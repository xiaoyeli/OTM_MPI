% Simulation speed is crucial when traffic simulation is used in real-time traffic management or traffic forecasting applications \cite{lee2002framework}.
\IEEEPARstart{R}{ecent} years have seen significant technological changes in transportation with the advent of shared-mobility like Uber and Lift, electric vehicles, and autonomous. There is a need for government agencies and industry to understand the impact of wide-spread adoption of these new technologies on the large-scale transportation system. This in turn requires tools to model the transportation system at the regional scale, requiring large computing and memory resources. Parallel computation and modern supercomputers provide the power and memory to handle such large-scale traffic simulations. They also enable running thousands of parallel simulations that can answer future what-if scenarios.
Traffic simulation developers have adopted parallel computation techniques for traffic simulation of large networks. On a multiprocessor shared-memory computer, parallel simulation takes advantage of the computer's many processors to execute tasks in parallel. In this case, the simulation speed-up is limited by the number of processors on the one computer. On a multiprocessor distributed-memory computer system, parallel simulation can use processors on multiple computers while communicating over a message passing system. This paper describes such an approach for a fluid-based traffic simulation models. We begin by reviewing previous efforts, most of which have focused on vehicle-based (microscopic and mesoscopic) models. 

% We should have a section that show when to use macroscopic simulation and why it is needed at large-scale too. 

\subsection{Related Works}

\begin{table*}[htbp]
\centering
\begin{tabular} { | l | p{25mm} | p{80mm}| } 
	\hline
	\hline
	\textbf{Software} & \textbf{Traffic Model} & \textbf{Parallel Method}\\ \hline
	Paramics \cite{cameron1996paramics} and FastTrans \cite{thulasidasan2009accelerating}    & Microscopic Model & Parallel computation on distributed-memory computer with MPI for inter-core communication \\ \hline
	TRANSIMS \cite{robertson1969transyt}     & Microscopic Model & Parallel computation on distributed-memory computer with MPI and PVM for inter-core communication \\ \hline
	AIMSUN \cite{ferrer1993aimsun2}, SEM-Sim \cite{aydt2013multi} and MEgaffic \cite{osogami2012research}     & Microscopic Model & Multi-threading \\ \hline
	SUMO \cite{behrisch2011sumo}    & Microscopic Model & Multi-threading for routing \\ \hline
	Dynemo \cite{nokel2002parallel}  & Mesoscopic Model & Network distribution through master-slave parallelization \\ \hline
	Mobiliti \cite{chan2018mobiliti}     & Mesoscopic Model & Parallel discrete event with GASNet \\ \hline
	POLARIS \cite{auld2016polaris}     & Mesoscopic Model & Multi-threading \\ \hline
	BEAM \cite{aboutBeam}    & Mesoscopic Model & Akka library \\ \hline
	\cite{xu2014mesoscopic,song2017supporting}and \cite{strippgen2009multi}     & Mesoscopic Model & Parallel computation on GPUs \\ \hline
	\cite{chronopoulos1998real}     & Macroscopic Model & Distribution of freeway sections on multiple processes of a nCubes2 distributed-memory parallel computer \\ \hline
	\cite{johnston1999parallelization}     & Macroscopic Model & Distribution of highway segments on multiple processes of a Cray T3W distributed-memory parallel computer \\ \hline
	OTM & Macroscopic Model & Parallel computing on modern mult-node computer systems with MPI for inter-core communication\\ \hline
%	OTM     & Macroscopic Model & Domain decomposition of network with highway, arterial and local roads. Use MPI for inter-process communication \\ \hline
\end{tabular}
\caption{Traffic Simulation Software}
\label{tab:softwares}
\end{table*}

%\xslnote{Define micro-, meso- and macro simulations.}

Since Greenshield presented the first traffic model in 1934 \cite{greenshields1934photographic}, three main traffic model categories have emerged: microscopic models, macroscopic models and mesoscopic models. Microscopic models traffic by representing the behavior and interaction of individual vehicles. Macroscopic models describe traffic as a continuum flow with density, flow and speed parameters. Mesoscopic models combine characteristics of macroscopic and microscopic models: traffic movement is represented in aggregate way, while the rules of behavior are modeled per individual vehicles \cite{van2015traffic}. Traffic simulation software tools can also be grouped into microscopic, macroscopic and mesoscopic simulators depending on the model they implement. 

Most prominent traffic simulation software that have parallel computation implementations are microscopic traffic simulators. Parallel Microscopic Simulation (Paramics) \cite{cameron1996paramics} microscopic simulator was implemented on a the T3D parallel computer with Message Passing Interface (MPI) for inter-processor communication. FastTrans \cite{thulasidasan2009accelerating} and TRansportation ANalysis and SIMulation System (TRANSIMS) \cite{nagel2001parallel} microscopic simulators use parallel computation on distributed-memory
%multi-core \xslnote{In 1996, there is no multi-core machine.} 
computer systems by dividing the road network into multiple sub-networks that are distributed among processes running on the multiple computing cores; FastTrans use MPI library to share boundary information among processes, while TRANSIMS can support
both MPI and Parallel Virtual Machine (PVM) communication interface. The Advanced Interactive Microscopic Simulator for Urban and Non-Urban Networks (AIMSUN) \cite{ferrer1993aimsun2}, Scalable Electro-Mobility Simulation (SEM-Sim) \cite{aydt2013multi}, and the IBM Mega Traffic Simulator (MEgaffic) \cite{osogami2012research} implement parallel computation through the multi-threading approach, which enables to run multiple threads in parallel to update simulated agents simultaneously. The Simulation of Urban Mobility (SUMO) microscopic simulator
exploits multi-threading parallel computation for vehicle routing,
but the rest of simulation is single-core \cite{behrisch2011sumo}.
The Behavior, Energy, Autonomy and Mobility (BEAM) simulator \cite{aboutBeam} extends the Multi-Agent Transportation simulator (MATSIM) \cite{horni2016multi}, which is a microscopic traffic simulator. BEAM incorporates parallel computation via the Akka \cite{akka} library, which implements the Actor Model computation \cite{actorModel}. 

 For mesoscopic traffic simulation, N{\"o}kel and Schmidt present a parallel implementation for the Dynemo mesoscopic simulator that uses a master-slave parallelization. The master process initializes the simulation and distributes the network among multiple slave processes. The slave processes calculates the spatial motion of the vehicles on their sub-networks and shares boundary information through PVM communication interface. Mobiliti \cite{chan2018mobiliti} is another mesoscopic traffic simulator that model links as agents and vehicles as events. It uses the parallel discrete event paradigm and a GASNet communication layer to enable parallel computation on high performance computing resources or clusters.The POLARIS mesoscopic simulator \cite{auld2016polaris} exploits parallel computation via multi-threading. References  \cite{xu2014mesoscopic,song2017supporting,strippgen2009multi} present mesoscopic traffic simulation frameworks that take advantage of parallel computation on graphics processing units (GPUs). 

On the other hand, parallel computation has not been applied extensively in macroscopic traffic simulation. In~\cite{chronopoulos1998real}, Chronopoulos and
Johnston present a parallel implementation of a continuum traffic flow model
for freeway modeling on a
nCubes2 hypercube distributed-memory parallel computer, where freeway is
partitioned into equal segments and assigned to different processors.
Each processor sends the calculated density, speed and volume on its segment to processors handling adjacent segments. Similarly, Johnston and Chronolopoulus \cite{johnston1999parallelization} describe a parallelization of lax-momentum traffic model for a highway that divided the highway into equal segments assigned to different processors. The simulation was run on Cray T3E 
%\xslnote{I know about T3D, T3E, but not T3W}
distributed-memory computer system. Though \cite{chronopoulos1998real} and \cite{johnston1999parallelization} use parallel computation for macroscopic traffic simulation on distributed-memory computers, they are focused on freeway simulation. In addition, the performed simulation studies were on small networks (18 to 20 mile freeway) and were executed on older computer systems.

The work we present here is based on the Open Traffic Models simultor. OTM is a full-featured traffic simulator that has in the past been used in single-process environments. OTM is a \textit{hybrid} simulator, in the sense that it incorporates multiple (macroscopic, mesoscopic, and microscopic) models which can be combined arbitrarily on a single network. It also supports multiple road geometries (turn pockets, managed lanes on freeways, etc.), multiple vehicle types, and a variety of traffic control systems (traffic signals, variable speed limits, onboard information systems, etc.). Here we focus on the parallelization of the macroscopic model only. This is both because there are many high quality distributed implementations of vehicle-based models, and because it avoid some of the complexities of asynchronous updates that occur in vehicle-based model.
